\section{Trabalhos Relacionados}

Um trabalho tangente \cite{b1}, focado tanto na aplicação dos modelos de \textit{Machine Learning} em \textit{batch} quanto em \textit{stream}, buscou comparar esses modelos de aprendizagem, uma vez que as variáveis utilizadas em um modelo variam com o tempo, prejudicando o desempenho do mesmo. Os modelos \textit{batch} utilizados foram: Regressão Logística, J48 (uma Árvore de Decisão), \textit{Naive Bayes} e \textit{Random Forest}. Com relação aos modelos em \textit{stream}, temos o \textit{Hoeffding Tree}, \textit{Hoeffding Adaptive Tree}, \textit{Leveraging Bagging} e \textit{Adaptive Random Forest}. Como resultados obtidos, para 2 dos 3 \textit{datasets} dos quais os experimentos foram conduzidos, os resultados obtidos pelos modelos em \textit{stream} foram comparáveis com os resultados dos modelos em \textit{batch}, que possuem um bom desempenho para a tarefa de análise de crédito.