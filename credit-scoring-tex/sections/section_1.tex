\section{Introdução}
A análise de crédito é uma prática comum entre as instituições financeiras e seguradoras, geralmente realizada por uma pessoa especializada, de maneira a decidir se um empréstimo será ou não fornecido para um indivíduo ou instituição. Essa pessoa levará em conta fatores subjetivos e objetivos, analisando o contexto da instituição, ou no caso, o histórico de um indivíduo, ponderando acerca da decisão \cite{b1}. Atualmente, existem muitas aplicações onde essa tarefa é delegada para algoritmos de \textit{Machine Learning} (pelo menos na maior parte dos casos), onde o modelo será treinado com dados envolvendo informações de renda, CEP, empregabilidade, entre outros. A criação de um modelo eficiente é complexa, uma vez que inúmeros fatores prejudicam esse objetivo. Na prática, geralmente os dados apresentam muitos valores faltantes, erros, ruídos, bem como desbalanceamento de classes (no caso existir muito mais registros de uma natureza em detrimento de outra nos dados). 

Como supracitado, a base utilizada nesse trabalho pertence a uma instituição financeira brasileira, e está devidamente anonimizada. Nosso objetivo é realizar a testagem de alguns modelos diferentes de Machine Learning, após os passos de pré processamento dos dados, seleção de atributos e, se necessário, normalização. Como principal métrica de validação, vamos utilizar o \textit{Kolmogorov-Smirnov} (KS), para avaliar o quão bem o modelo separa as duas classes, respectivamente adimplentes (0) e inadimplentes (1).

Este documento está organizado segundo as seções: ``II. Análise de Crédito e \textit{Machine Learning}", onde discutiremos alguns conceitos fundamentais relacionados ao problema em questão; ``III. Trabalhos Relacionados", onde apresentaremos trabalhos semelhantes e discutiremos acerca de suas metodologias e resultados; ``IV. Protocolo Experimental", em que apresentaremos a forma como os experimentos foram conduzidos, modelos utilizados, bibliotecas, métricas, seleção de atributos, parâmetros dos modelos, estratégias de validação e base de dados; ``V. Resultados", onde apresentaremos os resultados dos modelos em validação e \textit{features} selecionadas por métodos de filtro e através dos modelos; ``VI. Discussão dos Resultados", em que discutiremos acerca dos resultados apresentados na seção anterior; ``VII. Conclusão", onde iremos dar a palavra final deste trabalho, bem como apresentar ideias para trabalhos futuros.