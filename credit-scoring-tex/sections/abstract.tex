\begin{abstract}
	A análise de crédito é uma prática comum entre as instituições financeiras e seguradoras, geralmente realizada por uma pessoa especializada, de maneira a decidir se um empréstimo será ou não fornecido para um indivíduo ou instituição. Atualmente, é comum também esse processo ser realizado por algoritmos, de maneira a automatizar essa decião, levando em contas fatores como renda, histórico entre outros. Uma tarefa de classificação do indivíduo ser um bom pagador (adimplente) em detrimento de não ser (inadimplente) é complexa, pois lidamos com fatores como desbalanceamento (mais registros de pessoas adimplentes), valores faltantes, uma maioria de pessoas com renda familiar abaixo de R\$ 6900,00, entre outros; esses fatores são um desafio para a construção de modelos eficientes. Neste trabalho utilizamos uma base real, anônima, de uma instituição financeira do Brasil, e conduzimos uma sequência de testes com modelos distintos. Ao fim, obtemos um KS de 27.1 em teste, utilizando o modelo XGBoost. Também obtemos um KS em teste de 26.9 para o modelo \textit{LightGBM}, e concluímos através do teste Wilcoxon que ambos modelos não apresentaram diferenças significativas para a tarefa de classificação nesse problema de análise de crédito.
\end{abstract}

\begin{IEEEkeywords}
	Análise de Crédito, Machine Learning, XGBoost, LightGBM
\end{IEEEkeywords}