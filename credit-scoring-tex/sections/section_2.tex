\section{Análise de Crédito e Machine Learning}

A análise de crédito é um processo muito importante entre as instituições financeiras e seguradoras, de maneira que esse processo visa reduzir o risco e as perdas oriundas de inadimplência (maus pagadores), ou sinistros em potencial \cite{b1}. Essa decisão, delegada a algoritmos de \textit{Machine Learning} ou modelos de forma geral, busca automatizar e reduzir de forma massiva esse gargalo, considerando uma maior acurácia de acerto e velocidade no processo. 

O \textit{Machine Learning} aplicado em problemas dessa natureza é uma questão de aprendizado supervisionado, mais especificamente, um problema de classificação. Temos os atributos (colunas) de dados de uma base, tais como renda média da casa do proponente, nível educacional, e até mesmo histórico de pagamentos e afins em alguns modelos; todas essas colunas (ou \textit{features}) são passadas no treinamento do nosso classificador, que na etapa de teste vai realizar uma classificação binária: ou o indivíduo é considerado adimplente (0), ou então inadimplente (1).